\begin{hcarentry}[updated]{Haskell XML Toolbox}
\label{hxt}
\report{Uwe Schmidt}
\status{fifth major release (current release: 6.1)}
\entry{updated}% done, 7.11.2006
\makeheader

\subsubsection*{Description}

The Haskell XML Toolbox is a collection of tools for processing XML with
Haskell. It is itself purely written in Haskell 98. The core component of the
Haskell XML Toolbox is a validating XML-Parser that supports
almost fully the Extensible Markup Language (XML) 1.0 (Second Edition),
There is validator based on DTDs and a new more powerful validator for
Relax NG schemas.

The Haskell XML Toolbox bases on the ideas of HaXml~\cref{haxml} and HXML,
but introduces a more general approach for processing XML with Haskell.
Since release 5.1 there is a new arrow interface similar to the approach
taken by HXML. This interface is more flexible than the old filter approach.
It is also safer, type checking of combinators becomes possible with the arrow
interface.

\subsubsection*{Features}

\begin{compactitem}
\item Validating XML parser
\item Very liberal HTML parser
\item XPath support
\item Full Unicode support
\item Support for XML namespaces
\item Flexible arrow interface with type classes for XML filter
\item Package support for ghc
\item Native Haskell support of HTTP 1.1 and FILE protocol
\item HTTP and access via other protocols via external program curl
\item Tested with W3C XML validation suite
\item Example programs for filter and arrow interface
\item Relax NG schema validator based on the arrows interface
\item A HXT Cookbook for using the toolbox and the arrow interface
\item Basic XSLT support (next release)
\end{compactitem}

\subsubsection*{Current Work}

A master thesis has been finished developing an XSLT
system. The reuslt is a rather complete implementation of
an XSLT transformer system. Only minor features are missing.
The implementaion consists of about only 2000 lines of Haskell code.
The XSLT module will be included in the next HXT release.

A second master studends project will be finished until end of 2006.
The title is {\em A Dynamic Webserver with Servlet Functionality in
  Haskell Representing all Internal Data by Means of XML}
HXT with the arrows interface has been used for all internal data processing.
The results of this work will be available with the next HXT release.

The next HXT release is planned for December 2006.

\FurtherReading

The Haskell XML Toolbox Web page
(\url{http://www.fh-wedel.de/~si/HXmlToolbox/index.html})
includes downloads, online API documentation, a cookbook with nontrivial examples
of XML processing using arrows and RDF documents, and master thesises describing the
design of the toolbox, the DTD validator and the arrow based Relax NG
validator.
A getting started tutorial about HXT is avaliable in the Haskell Wiki (\url{http://www.haskell.org/haskellwiki/HXT}).
\end{hcarentry}
