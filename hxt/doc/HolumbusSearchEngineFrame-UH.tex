% HolumbusSearchEngineFrame-UH.tex
\begin{hcarentry}[updated]{Holumbus Search Engine Framework}
\label{holumbus}
\report{Uwe Schmidt}%05/12
\participants{Timo~B.~Kranz, Sebastian Gauck,
  Stefan Schmidt}
\status{first release}
\makeheader

\subsubsection*{Description}

The Holumbus framework consists of a set of modules and tools
for creating fast, flexible, and highly customizable search engines with Haskell.
The framework consists of two main parts. The first part is the indexer for extracting the data
of a given type of documents, e.g., documents of a web site, and store it in an appropriate index.
The second part is the search engine for querying the index.

An instance of the Holumbus framework is the Haskell API search engine Hayoo!\
(\url{http://holumbus.fh-wedel.de/hayoo/}).

The framework supports distributed computations for building indexes
and searching indexes. This is done with a MapReduce like framework.
The MapReduce framework is independent of the index- and
search-components, so it can be used to develop distributed systems
with Haskell.

The framework is now separated into four packages, all available on
Hackage.

\begin{compactitem}
\item The Holumbus Search Engine 
\item The Holumbus Distribution Library
\item The Holumbus Storage System
\item The Holumbus MapReduce Framework
\end{compactitem}

The search engine package includes the indexer and search modules,
the MapReduce package bundles the distributed MapReduce system.
This is based on two other packages, which may be useful for their on:
The Distributed Library with a message passing communication layer
and a distributed storage system.

\subsubsection*{Features}

\begin{compactitem}
\item Highly configurable crawler module for flexible indexing of structured data
\item Customizable index structure for an effective search
\item {\em find as you type} search
\item Suggestions
\item Fuzzy queries
\item Customizable result ranking
\item Index structure designed for distributed search
\item Git repository containing the current development version of all packages under
  \url{https://github.com/fortytools/holumbus}
\item Distributed building of search indexes
\end{compactitem}

\subsubsection*{Current Work}

Currently there are activities to optimize the index structures
of the framework. In the past there have been problems with the
space requirements during indexing. The data structures and evaluation
strategies have been optimized to prevent space leaks. A second index
structure working with cryptographic keys for document identifiers
is under construction. This will further simplify partial indexing and
merging of indexes.

The second project, a specialized search engine for the FH-Wedel web site has been finished
 \url{http://w3w.fh-wedel.de/}. The new aspect in this application is a specialized
free text search for appointments, deadlines,
announcements, meetings and other dates.

The Hayoo! and the FH-Wedel search engine have been adopted to run on top of the
Snap framework \url{http://snapframework.com}.

\FurtherReading

The Holumbus web page
(\url{http://holumbus.fh-wedel.de/})
includes downloads, Git web interface, current status, requirements, 
and documentation.
Timo Kranz's master thesis describing the Holumbus index structure and
the search engine is available at
\url{http://holumbus.fh-wedel.de/branches/develop/doc/thesis-searching.pdf}.
Sebastian Gauck's  thesis dealing with the crawler component is
available at
\url{http://holumbus.fh-wedel.de/src/doc/thesis-indexing.pdf}
The thesis of Stefan Schmidt describing the Holumbus MapReduce is
available via \url{http://holumbus.fh-wedel.de/src/doc/thesis-mapreduce.pdf}.
\end{hcarentry}
